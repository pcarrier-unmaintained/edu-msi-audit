\section{Outils existants}
\subsection{Vue d'ensemble}
\label{general}

Nous avons constaté des problèmes récurrents et globaux. Ces problèmes sont pour
la plupart identifié de longue date par l'équipe de direction et le bureau. Nous
ne les indiquerons pas pour chaque sous-système.

\begin{enumerate}
\item \textbf{La faible utilisation d'outils numériques.} Les opérateurs manipulent et archivent
      trop de documents papier.
      L'organisation de leurs archives est généralement insufisamment documentée.
      Nous recommandons de largement numériser les documents papier dans un premier temps,
      de déterminer la pertinence de l'extraction des différentes données qu'ils contiennent,
      notamment à des fins statistiques, afin d'établir des procédures de saisie.
\item \textbf{La redondance des informations,} notamment sur les usagers, qu'il s'agisse d'individus
      (bénéficiaires des assistances API, ENE, clients au Grand Café dont salariés et LocaMIPE)
      qui peuvent être adhérents ou d'associations qui peuvent être adhérentes ADAM.
      Nous recommandons d'établir une base de données centralisée, et de faciliter l'identification
      des adhérents à l'aide de leur carte d'adhérent (qui serait équipée au minimum d'un code barre),
      et/ou à terme l'identification des étudiants à l'aide de leur carte multiservices
      (aujourd'hui seulement utilisée par le CROUS et l'INPG, à terme probablement
      par les Universités, éventuellement par la TAG) qui pourrait alors se substituer
      pour ces publics à la carte d'adhérent.
      De plus, les sous-systèmes doivent s'interfacer. Le comptable doit notamment transférer
      beaucoup d'informations des sous-systèmes de services au sous-système comptable.
\item \textbf{Le manque d'ouverture sur l'environnement.} De nombreuses informations, pour certaines
      connues et diffusées au sein du bâtiment, ne sont pas disponibles sur le site Internet.
\item \textbf{Le manque de retour qualité.} Si d'expérience, la structure est ouverte aux critiques
      et suggestions, il n'existe pas de moyens autres qu'oraux pour les communiquer.
      Chaque service gagnerait à mener des enquêtes en la matière, et le système de gestion
      à disposer d'informations complètes comme synthétiques.
\item \textbf{Le manque d'outils d'aide à la décision.} Même quand les informations sont correctement
      intégrées aux outils numériques, la génération de statistiques est absente ou
      incomplète.
\item \textbf{La disparité des systèmes de sauvegarde}, leur caractère manuel et/ou insuffisant
      dans de nombreux cas.
      Nous recommandons l'utilisation d'un système centralisé pour simplifier la maintenance,
      la surveillance \textit{via} des commandes d'alerte et la multiplicité de la redondance.
\end{enumerate}

\subsection{Sous-systèmes}

\subsubsection{Gestion commerciale, gestion administrative}

Pour sa gestion commerciale et administrative, la direction utilise un logiciel spécialisé
 édité par la société EBP.
Il lui permet de gérer sa comptabilité, la facturation et les devis ainsi que la
communication bancaire. Cependant, la rédaction bilans comptables annuels sont externalisés.

\subsubsection{Gestion du personnel}

La gestion du personnel est externalisée à une entreprise tierce.
L'association Éponyme n'a par conséquent aucun outil dédié.

\subsubsection{Adhérents et multidiffusion}
\label{multidiffusion}

Un poste consacré aux adhésions est installé au comptoir du Grand Café.
L'accès matériel permet uniquement l'utilisation d'un logiciel Web, conçu par un des auteurs.
Il permet la saisie par l'adhérent de ses coordonnées, et au serveur de confirmer
le règlement sur place (caisse séparée de l'activité de café) ou lors de l'inscription
à l'Université (proposé à l'UPMF et à Stendhal).

Après une phase d'identification, le comptable peut accéder à un historique des inscriptions
par jour et semaine. Nous préconisons qu'il lui soit automatiquement transmis par E-mail
à la fin de chaque période.

Après une phase d'identification, le responsable du service ENE peut accéder à une liste
des adhérents sous forme de tableau ou au format CSV. Il l'utilise entre autres pour en
extraire les courriels qu'il inscrit par la suite sur une liste de multidiffusion.
Nous préconisons que l'inscription soit automatisée
et que ces tableaux évoluent dans un outil complet
dans le cadre de la centralisation abordée en \ref{general}.

La multidiffusion utilise un système administré par une association partenaire.
Elle pose de sérieux problèmes d'ergonomie et n'est pas ouverte à l'automatisation de
la gestion des inscriptions. Cependant, un des auteurs a utilisé l'ingénerie inverse
et Éponyme peut donc désormais automatiser certaines tâches. Il est notable que les années
précédentes, les milliers d'adresses électroniques des adhérents étaient saisies
individuellement au cours de l'année individuellement dans une interface Web.
Nous recommandons donc de déployer une solution technique plus ouverte
sur le serveur d'Éponyme.

\subsubsection{Grand Café}

Les serveurs du Grand Café et le comptable utilisent le logiciel Orchestra
Point de Vente Bar-Restaurant, de la société Orchestra Software.

Tout est contrôlé depuis un ordinateur au Grand Café. Y sont gérés les
références produits (environ 70 cette année), les stocks, les tarifications
(3 modes : plein tarif, tarif réduit et tarif salariés ; le plein tarif
est appliqué en journée pour les non-adhérents et en soirée pour les boissons
alcoolisées, le tarif réduit est appliqué en journée pour les adhérents
et en soirée pour le reste), les mouvements et les comptes clients (pour les
salariés).

Les références produits et tarifications associées sont gérées par le comptable.

L'état des stocks est confronté à un inventaire mensuel établi par une équipe
tournante parmi les serveurs, et mis à jour par les serveurs à chaque
livraison.

Depuis cette année, les comptes client doivent rester strictement positifs
(plus d'«~ardoise~»).
Ils sont alimentés par dépôt de liquide par les salariés auprès des serveurs.
Un rapport de solde du compte du salarié est édité à chaque mouvement.

Les «~X~» et «~Z~» sont des rapports sur les mouvements d'une période et
l'état de la caisse. Le Z initialise une nouvelle période.
Le premier est utilisé au cours de la journée par les serveurs, en début et fin
de service. Il est comparé avec un compte manuel de la monnaie et des billets,
reporté sur un document papier transmis en fin de journée à la direction.
Le second est utilisé par le comptable en début de journée.

\subsubsection{Assistance informatique}

L'outil Assistance informatique est utilisé par l'ENE pour l'activité éponyme.

Il a plusieurs vocations :
\begin{itemize}
\item Suivre les assistances en cours, que ce soit par téléphone,
      courriel ou sur place, notamment en cas de changement d'opérateur ;
\item Remonter des statistiques au système de direction, par exemple les temps
      moyens de traitement, y compris par opérateur, le nombre de traitements,
      etc.
\item Remonter des statistiques aux partenaires publics sur les problèmes
      rencontrés par les usagers dans l'utilisation de leurs outils, notamment
      pour le Bureau Virtuel ;
\item Constituer une base de connaissances centralisée en standardisant les
      procédures d'assistance.
\end{itemize}

Il s'agit d'une application Web développée en interne quand le service était
assuré directement par la Direction des Systèmes d'Information de Grenoble,
Université de l'Innovation, et sur l'installation de laquelle EVE n'a et
n'aura aucun contrôle direct.

Son interface est partiellement présentée en annexe \ref{gestion_incidents}.

Nous préconisons donc à Éponyme de s'assurer de la pérennité de cet outil
en obtenant les données historiques et en l'installant sur son serveur.

\subsubsection{LocaMIPE}

La base de données de LocaMIPE est utilisée par l'ENE pour le suivi des
locations des postes du parc dédié et leur vente en fin de cycle.

Nous ne nous attarderons pas sur le logiciel y étant actuellement associé,
puisqu'il s'agit du répandu phpMyAdmin. L'inadaptation totale de ce logiciel
pose de sérieux problèmes.

Nous préconisons donc le développement d'un logiciel adapté. À défaut
de s'intégrer dans un SI global comme développé en \ref{general}, il
devra au minimum identifier les usagers entre leurs locations et achats
(entre autres pour connaître le nombre réel d'usagers, le nombre de passages
par usager).

\begin{itemize}
\item La génération des contrats et factures (aujourd'hui écrits manuscrits
      par le client) ;
\item La génération de statistiques temporelles, notamment sur l'utilisation
      des services selon les horaires hebdomadaires d'ouvertures et
      la période de l'année ;
\item Le suivi des locations en retard et des défauts de paiement, de l'état
      du parc, des dysfonctionnements des machines ;
\item La communication sur le site de EVE de l'état et des prévisions de
      stock et la réservation à distance ;
\item Le retour de satisfaction et les requêtes des clients sur la suite
      logicielle, les interactions avec le service.
\end{itemize}

Il disposera de profils adaptés à l'équipe de direction, aux opérateurs et
aux usagers et prévoiera une grande souplesse des \textit{scenarii} étant
donnés les imprévus liés aux difficultés des populations de clients.

\subsubsection{Pépinière}

Le SI de la Pépinière est en pleine mutation. L'un des auteurs coordone
la conception d'un logiciel intégré dans un nouveau site de EVE, dont une version
est aujourd'hui en phase d'expérimentation, une importante mise à jour
rétrocompatible en cours de développement (son cahier des charges étant basé
sur les premiers retours).

Il permet la constitution d'un annuaire des associations du campus, qu'elles
soient membres ou non d'ADAM. Ces dernières peuvent actualiser leur fiche,
et ADAM modère toute opération. Cette annuaire fait partie des missions définies
dans la première comme la deuxième DSP, mais n'a pu être mis en place jusqu'ici
puisque les solutions envisagées les années précédentes étaient trop coûteuses
en temps opérateur.

ADAM peut référencer le matériel et les salles disponibles aux associations,
les associations peuvent consulter les calendriers de leur disponibilité,
saisir des dossiers d'organisation d'événements en indiquant leurs besoins,
le COMA remettre ses décisions (voir annexe \ref{demande_coma}).

Les visiteurs peuvent consulter le calendrier des événements.

\subsubsection{API}

Le «~Guide des étudiants étrangers~» est accessible en ligne après saisie par
l'étudiant de ses coordonnées et vérification de son adresse de courriel (pour
inscription dans une liste de multidiffusion, voir \ref{multidiffusion}).

Toutes les assistances sont saisies avec des informations (dont la nationalité)
sur l'étudiant. De nouveau, phpMyAdmin est utilisé.

Nous préconisons donc le développement d'un logiciel adapté. À défaut
de s'intégrer dans un SI global comme développé en \ref{general}, il
devra au minimum identifier les usagers entre leurs demandes d'assistance
(entre autres pour connaître le nombre réel d'usagers, le nombre de passages
par usager).

\subsection{Multithèque}

La multithèque utilise un Système Intégré de Gestion de Bibliothèque (SIGB) libre, Koha3,
adapté en interne à la diversité des types de ressources disponibles (voir \ref{multitheque}).
Ce dernier permet aux opérateurs d'enregistrer les emprunts, notifie les usagers
par courriel après une semaine d'emprunt, notifie les opérateurs des retards pour permettre l'encaissement de la caution.

Il permet de consulter le catalogue (disponibilité comprise) à distance.

Nous suggérons de proposer la réservation des ressources à distance.

\subsubsection{Documentation}

L'association utilise plusieurs wiki pour sa documentation interne, et d'espaces FTP
communs créés selon les groupes de travail (dont un par service). Nous suggérons d'ouvrir
ce deuxième type d'outils à travailler l'utilisation d'une interface Web de partage
d'arborescence de documents.