\section{Outils existants}
\subsection{Vue d'ensemble}



\subsection{Sous-systèmes}

\subsubsection{EBP}

\subsubsection{Ressources humaines}

\subsubsection{Adhérents et multidiffusion}

\subsubsection{Grand Café}

Les serveurs du Grand Café et le comptable utilisent le logiciel Orchestra
Point de Vente Bar-Restaurant, de la société Orchestra Software.

Tout est contrôlé depuis un ordinateur au Grand Café. Y sont gérés les
références produits (environ 70 cette année), les stocks, les tarifications
(3 modes : plein tarif, tarif réduit et tarif salariés ; le plein tarif
est appliqué en journée pour les non-adhérents et en soirée pour les boissons
alcoolisées, le tarif réduit est appliqué en journée pour les adhérents
et en soirée pour le reste), les mouvements et les comptes clients (pour les
salariés).

Les références produits et tarifications associées sont gérées par le comptable.

L'état des stocks est confronté à un inventaire mensuel établi par une équipe
tournante parmi les serveurs, et mis à jour par les serveurs à chaque
livraison.

Depuis cette année, les comptes client doivent rester strictement positifs
(plus d'«~ardoise~»).
Ils sont alimentés par dépôt de liquide par les salariés auprès des serveurs.
Un rapport de solde du compte du salarié est édité à chaque mouvement.

Les «~X~» et «~Z~» sont des rapports sur les mouvements d'une période et
l'état de la caisse. Le Z initialise une nouvelle période.
Le premier est utilisé au cours de la journée par les serveurs, en début et fin
de service. Il est comparé avec un compte manuel de la monnaie et des billets,
reporté sur un document papier transmis en fin de journée à la direction.
Le second est utilisé par le comptable en début de journée.

\subsubsection{Assistance informatique}

L'outil Assistance informatique est utilisé par l'ENE pour l'activité éponyme.

Il a plusieurs vocations :
\begin{itemize}
\item Suivre les assistances en cours, que ce soit par téléphone,
      courriel ou sur place, notamment en cas de changement d'opérateur ;
\item Remonter des statistiques au système de direction, par exemple les temps
      moyens de traitement, y compris par opérateur, le nombre de traitements,
      etc.
\item Remonter des statistiques aux partenaires publics sur les problèmes
      rencontrés par les usagers dans l'utilisation de leurs outils, notamment
      pour le Bureau Virtuel ;
\item Constituer une base de connaissances centralisée en standardisant les
      procédures d'assistance.
\end{itemize}

Il s'agit d'une application Web développée en interne quand le service était
assuré directement par la Direction des Systèmes d'Information de Grenoble,
Université de l'Innovation, et sur l'installation de laquelle EVE n'a et
n'aura aucun contrôle direct.

Son interface est partiellement présentée en annexe \ref{gestion_incidents}.

Nous préconisons donc à Éponyme de s'assurer de la pérennité de cet outil
en obtenant les données historiques et en l'installant sur son serveur,
et de l'intégrer à un SI centralisé selon les règles définies en
\ref{integration_identites}.
TODO \label{integration_identites}

\subsubsection{LocaMIPE}

La base de données de LocaMIPE est utilisée par l'ENE pour le suivi des
locations des postes du parc dédié et leur vente en fin de cycle.

Nous ne nous attarderons pas sur le logiciel y étant actuellement associé,
puisqu'il s'agit du répandu phpMyAdmin. L'inadaptation totale de ce logiciel
pose de sérieux problèmes.

Nous préconisons donc le développement d'un logiciel adapté, centralisé selon
les règles définies en \ref{integration_identites}, dissociant donc les
utilisateurs de leurs emprunts. Il permettra :
\begin{itemize}
\item La génération des contrats et factures (aujourd'hui écrits manuscrits
      par le client) ;
\item La génération de statistiques temporelles, notamment sur l'utilisation
      des services selon les horaires hebdomadaires d'ouvertures et
      la période de l'année ;
\item Le suivi des emprunts en retard et des défauts de paiement, de l'état
      du parc, des dysfonctionnements des machines ;
\item La communication sur le site de EVE de l'état et des prévisions de
      stock et la réservation à distance ;
\item Le retour de satisfaction et les requêtes des clients sur la suite
      logicielle, les interactions avec le service.
\end{itemize}

Il disposera de profils adaptés à l'équipe de direction, aux opérateurs et
aux usagers et prévoiera une grande souplesse des \textit{scenarii} étant
donnés les imprévus liés aux difficultés des populations de clients.

\subsubsection{Pépinière}

Assos, matos, salles, demandes, événements

\subsubsection{API}

\subsubsection{Documentation}
