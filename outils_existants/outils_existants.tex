\section{Outils existants}
\subsection{Vue d'ensemble}
\subsection{Sous-syst�mes}
\subsubsection{Comptabilit�}
\subsubsection{Ressources humaines}
\subsubsection{Adh�rents et multidiffusion}
\subsubsection{Grand Caf�}
\subsubsection{Assistance informatique}
L'assistance informatique, situ�e dans le bureau de l'ENE (epsace num�rique �tudiant), dispose d'un outil de gestion des incidents. Il s'agit d'une application web d�velopp�e spc�cialement pour le service.

Interfa�ant une base de donn�es MySQL, le but de cette application est de surveiller l'activit� du service, notament de faire des statistiques sur les 
probl�mes les plus rencontr�s par les utilisateurs, par configuration (environnement logiciel et mat�riel), ainsi que le temps pass� par l'usager
dans les locaux.

Cet outil permet �galement de saisir la d�marche suivie pour r�soudre le probl�me afin de garder trace des mode op�ratoire �fficaces.

\subsubsection{LocaMIPE}
\subsubsection{P�pini�re}
Assos, matos, salles, demandes, �v�nements
\subsubsection{API ?}
\subsubsection{Documentation}
