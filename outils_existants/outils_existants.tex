\section{Outils existants}
\subsection{Vue d'ensemble}
\subsection{Sous-systèmes}
\subsubsection{Comptabilité}
\subsubsection{Ressources humaines}
\subsubsection{Adhérents et multidiffusion}
\subsubsection{Grand Café}
\subsubsection{Assistance informatique}
L'assistance informatique, située dans le bureau de l'ENE (epsace numérique étudiant), dispose d'un outil de gestion des incidents. Il s'agit d'une application web développée spcécialement pour le service.

Interfaçant une base de données MySQL, le but de cette application est de surveiller l'activité du service, notament de faire des statistiques sur les 
problèmes les plus rencontrés par les utilisateurs, par configuration (environnement logiciel et matériel), ainsi que le temps passé par l'usager
dans les locaux.

Cet outil permet également de saisir la démarche suivie pour résoudre le problème afin de garder trace des mode opératoire éfficaces. L'illustration de son interface est donnée en annexe \ref{gestion_incidents}, p.\pageref{gestion_incidents}.

\subsubsection{LocaMIPE}
\subsubsection{Pépinière}
Assos, matos, salles, demandes, événements
\subsubsection{API ?}
\subsubsection{Documentation}
