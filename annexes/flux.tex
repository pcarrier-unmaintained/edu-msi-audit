\subsubsection{Chaîne de fin de stock au Grand Café}

\begin{center}
\begin{dot2tex}
digraph FinStockBar {
  size = "7!"
  rankdir = "LR"
  node [ shape = "circle", style = filled, fillcolor=lightgray ];
  SI [ shape = "octagon" ];
  F [ shape = "doublecircle" ];
  SI -> S [ label = "1" ] ;
  S -> R [ label = "2" ];
  R -> F [ label = "3" ];
  F -> S [ label = "4" ];
  S -> SI [ label = "5" ];
  S -> C [ label = "6" ];
  C -> SI [ label = "7,9" ];
  C -> F [ label = "8" ];
}
\end{dot2tex}
\end{center}
\begin{enumerate}
\item Le SI (caisse du Grand Café) indique au S (serveur) une fin de stock ;
\item S transmet l'information au responsable stocks (R) ;
\item R passe une commande au fournisseur (F) ;
\item F livre le produit, fournit un bon de livraison et une facture au S ;
\item S met à jour le stock dans le SI ;
\item S transmet le bon de livraison et la facture au comptable (C) ;
\item S saisit la facture dans le SI ;
\item S effectue le règlement (8) auprès du F ;
\item S l'indique au SI.
\end{enumerate}

\subsubsection{Adhésion d'une association à la Pépinière}

Si l'association ne souhaite qu'être référencée sur le site de EVE,
la chaîne s'interrompt à l'étape \ref{stop_point}.

\begin{center}
\begin{dot2tex}
digraph AdhesionAsso {
  size = "7!"
  rankdir = "LR"
  node [ shape = "circle", style = filled, fillcolor=lightgray ];
  A [ shape = "doublecircle" ];
  SI [ shape = "octagon" ];
  C;
  P;
  A -> SI [ label = "1" ];
  SI -> P [ label = "2" ];
  P -> SI [ label = "3,6" ];
  SI -> A [ label = "4,7,10" ];
  A -> P  [ label = "5" ];
  P -> C  [ label = "8" ];
  C -> SI [ label = "9" ];
}
\end{dot2tex}
\end{center}

\begin{enumerate}
\item L'association (A) renseigne les informations nécessaires dans le SI ;
\item Le SI transmet à la Pépinière (P) la demande d'adhésion ;
\item P accepte la publication des données de l'association dans l'annuaire ;
\item Le SI envoie un mail à A l'informant de cette décision ;
\label{stop_point}
\item A transmet par chèque le règlement à P ;
\item P renseigne dans le SI le statut «~adhérente~» de A ;
\item Le SI envoie un mail à A l'informant de cette action ;
\item P transmet le chèque à la comptabilité (C) ;
\item C enregistre le paiement dans le SI ;
\item le SI envoie à A le reçu de ce paiement.
\end{enumerate}

\subsubsection{Demande de mise à disposition de salle ou matériel}

%edge [arrowsize=3.0]

\begin{center}
\begin{dot2tex}[circo]
digraph AdhesionAsso{
	rankdir = "LR"
	size = "9!"
	node [ shape = "circle", style = filled, fillcolor=lightgray ];
	A [ shape = "doublecircle" ];
	SI [ shape = "octagon" ];
	Po [ shape = "doublecircle" ];
	A-> SI [ label = "a"] ;
	SI -> P [ label = "b" ];
	SI -> Co [ headlabel = "c" ];
	SI -> Po [ label = "d" ];
	SI -> Ca [ label = "e" ];
	SI -> I [ label = "f" ];
	P-> SI [ label = "g" ];
	SI-> A [ label = "h" ];
	SI-> C [label = "i" ];
	C-> SI [ label = "j" ];
	Po -> SI [ label = "k" ];
}
\end{dot2tex}
\end{center}

L'association doit être adhérente. Cependant, elle peut n'adhérer qu'après approbation
de l'événement par le COMA.

\begin{enumerate}
\item L'association (A) renseigne les informations nécessaires dans SI [a] ;
\item Le SI transmet la demande à la Pépinière (P) [b] ;
\item Phase de validation du dossier par P, en boucle : 
	\begin{enumerate}
	\item P transmet à A les modifications à apporter au dossier via le SI [g,h] ;
	\item A transmet à P le dossier modifié via le SI [a,b] ;
	\end{enumerate}
\item La Pépinière valide le dossier dans le SI [g] ; 
\item Le SI transmet le dossier à la Commission Animation (C) [i] ;
\item C informe le SI de sa décision [j] ;
\item
\begin{itemize}
	\item \textbf{En cas d'acceptation :}
	\begin{enumerate}
		\item Le SI informe A de la décision de C [h] ;
		\item \textbf{Si l'événement se déroule le soir} :
		\begin{enumerate}
			\item \textbf{Selon les conditions de sécurité (type d'événement, jauge) :}
			\begin{enumerate}
				\item C indique au SI le besoin d'une équipe [j] ;
				\item Le SI le transmet au comptable (Co) [c] ;
				\item Le SI transmet à Pro One (PO) les horaires [d] ;
			\end{enumerate}
			\item Le SI indique au Café (Ca) les horaires [e] ;
		\end{enumerate}
		\item \textbf{Si A indique nécessiter un ingénieur son ou lumière} :
		\begin{enumerate}
			\item Le SI en informe le ou les ingénieurs (I) [f] ;
			\item Le SI en informe Co [c] ;
		\end{enumerate}
	\end{enumerate}
	\item \textbf{En cas d'acceptation sous condition :}
	              le SI informe A de la décision de C [h]. Retour en~1 ;
	\item \textbf{En cas de refus :} le SI informe A de la décision de C [h] ;
	\item \textbf{En cas de modification (annulation, réévaluation du public attendu :}
	\begin{enumerate}
		\item A indique au SI le changement [a] ;
		\item SI envoie une alerte à P [b] ;
		\item \textbf{Si P juge que le besoin d'une équipe de sécurité évolue :}
		\begin{enumerate}
			\item P indique au SI le besoin d'une équipe [g] ;
			\item Le SI le transmet à Co [c] ;
			\item Le SI le transmet à PO [d] ;
			\item PO notifie le SI de la possibilité de changement [k] ;
			\item Le SI le transmet à Co [c] ;
			\item Le SI le transmet à P [b].
		\end{enumerate}
	\end{enumerate}

\end{itemize}
\end{enumerate}