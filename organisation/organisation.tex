\section{L'organisation}

\subsection{Présentation générale}
\label{presentation}

Éponyme est la seule association étudiante interuniversitaire en France
ayant la gestion d'un bâtiment universitaire, l'Espace Vie Étudiante.
Celui-ci est mis à sa disposition par l'Université Pierre Mendès-France,
délégataire au nom des membres de Grenoble, Université de l'Innovation.
Avec plus de 3000 membres en 2008-2009, il s'agit également de la plus
importante association étudiante locale. Elle constitue un modèle
d'implication étudiante en France, avec le réseau du Grand Dijon.

Il s'agit, depuis sa construction achevée en 2002, d'un lieu-clef de
l'animation culturelle et festive du campus. Il accueille un projet né
dans les années 1980, travaillé avec les universités dès 1992 et porté à
partir de 1997 par une vaste majorité des associations étudiantes,
réunies sous le drapeau «~Les rÊVEurs~», auprès des acteurs publics de
l'agglomération.

Avec d'ores et déjà plus de 3000 membres en 2008-2009, elle rencontre comme
nombre d'associations sans profonde vocation commerciale de cette ampleur des
difficultés organisationnelles entre adhérents, bureau, salariés et
financeurs.

L'obsolescence des statuts, l'absence de vision à long terme du devenir de l'espace
par les fondateurs et des problèmes de communication sont à l'origine d'une
période charnielle où beaucoup reste à construire et où les besoins de
refonte du mode de gestion, de standardisation des procédés et de redéfinition
des objectifs et des moyens y étant accordés se font fortement ressentir.

La croissance de la structure l'amène également à adopter des outils
o\-pé\-ra\-tion\-nels plus adaptés à la nouvelle échelle des traitements, pour
l'instant exclusivement conçus sur commande et en interne étant données ses
spécificités mais également la disponibilité d'équipes compétentes en la
matière et les faibles moyens financiers allouables.

Après plusieurs mois de négociations liées au renouvellement de la délégation
de service public, un groupe de travail a notamment été créé courant novembre
2008 pour réécrire les statuts et établir un réglement intérieur.

Nous nous attacherons donc, après un survol des contraintes liées aux contrats
et conventions impliquant l'association, à décrire et critiquer la
structuration actuelle, à la mettre en perspective au regard des propositions
établies dans le chantier de réforme.

\subsection{Contrats et conventions}
\label{contrats}

\subsubsection{Délégation de Service Public}

Il s'agit du principal mode de subventionnement des activités d'Éponyme :
3,25 euros sont versés annuellement par l'Université Joseph Fourier,
l'Université Pierre Mendès-France, l'Université Stendhal et Grenoble INP
(qui a quitté le statut d'Université pour celui de Grand Établissement
 au cours de la première convention) pour chaque étudiant y étant inscrit.
Ce montant est issu d'une réevaluation des frais de fonctionnement de EVE,
et s'élevait précédemment à 3,00 euros.

Pour information, un audit réalisé par les services financiers de Grenoble,
Université de l'Innovation a quantifié en 2007 les besoins à 3,78 euros.
Éponyme doit ainsi rapidement diversifier ses sources de financement.

Le nouveau bureau a pris ses fonctions au début des discussions autour de
son renouvellement. Son principal objectif était alors de pérenniser des
services que sont API, présenté en \ref{api}, et l'ENE, présenté en \ref{ene},
en les faisant inscrire dans la nouvelle
convention. Devant l'échec de cette démarche en ce qui concerne l'ENE, une
subvention a été demandé au CESR. En cas de refus, le service pourrait être
fermé début 2008.

La convention est fournie en annexe sous le titre
«~Projet de gestion du bâtiment EVE~».

TODO Conventions ENE

\subsection{Structuration}
\label{structuration}

L'organigramme que nous avons produit est fourni en annexe sous le titre
«~Organigramme de l'association Éponyme~». Les multiples intrications et la
profondeur de la structure nous empêchent d'en produire une version complète
et sans duplication.

L'organisation utilise l'équivalent de regroupements par fonction (par exemple
la Direction qui regroupe les salariés en charge de la gestion administrative)
et par marché (par exemple la commission Pépinière regroupant tous les services
aux associations, les autres commissions se focalisant sur les publics
individuels).

Cet organigramme est complété par un parcours de l'ensemble des instances en
annexe \ref{parcoursstruct}. Y sont développés leurs rôles, moyens,
composition et mode de fonctionnement.

\label{api}
\label{ene}
