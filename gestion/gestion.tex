\section{Règles de gestion}
\label{gestion}

\subsection{Réunions}

Il existe plusieurs types de réunions régulières, aux membres définis,
mais où les rôles (hors celui de la présidence et cas du CA),
et les problématiques traitées s'adaptent en majorité aux besoins ponctuels.

Cependant, il convient dans la contextualisation du SI de préciser quels types
de décisions sont amenées à y être prises afin d'en déduire quels
indicateurs apporter pour les éclairer.

Pour l'instant, seuls sont utilisés les indicateurs mis en place dans le cadre
de la mise en conformité avec la Loi Organique relative aux Lois de Finance (LOLF),
principalement la comptabilisation des traitements.

\paragraph{Conseil d'Administration} Toutes les six semaines environ.
Voir \ref{ca}.

\paragraph{Bureau} Chaque semaine. Voir \ref{bureau}.

\paragraph{Réunion salariale} Un lundi sur 3, discussion entre l'ensemble
des salariés et le bureau pour éclairer ce dernier sur ses
décisions stratégiques.

\paragraph{Réunion des services} Deux lundi sur 3, discussion des
salariés du service (séparemment ADAM, ENE et API) avec le directeur
afin d'éclairer la direction sur ses décisions tactiques.

\paragraph{Réunion serveurs} Tous les mois, les serveurs et les responsables
du Grand Café se réunissent pour la diffusion d'informations générales,
la remontée de problèmes, l'organisation, etc. Des formations sont
assurées en début d'année.

\paragraph{Réunion des associations conventionnés} Trois fois par an,
discussion entre les associations conventionnées (listées en \ref{ca}),
le bureau et la direction autour du fonctionnement général du bâtiment,
de l'évolution des projets communs, etc.

\paragraph{Réunion des responsables de commissions} Chaque trimestre,
discussion entre les responsables des commissions et le bureau
autour de la répartition du budget consacré aux événements des commissions,
de la coordination des activités, des problématiques rencontrées dans
l'organisation des équipes bénévoles.

À ces réunions s'ajoutent les présentations annuelles de bilan et perspective
de l'association auprès des financeurs.

\subsection{Multithèque}
\label{multitheque}

Les règles de gestion de la multithèque n'étant communes avec toutes les
organisations de ce type, nous avons souhaité les développer.

L'accès au service est réservé aux adhérents d'Éponyme. Les locaux sont ouverts
lors de permanences bénévoles, de 12h à 14h du lundi au jeudi.

Pour pouvoir emprunter, ils doivent fournir une photocopie de justificatif de domicile,
une photocopie de pièce d'identité et un chèque de caution de 50~{\texteuro}.

Ils peuvent alors emprunter un support pour une durée maximale de 2 semaines.
En cas de retard, leur chèque est encaissé. En cas de retour, les 50~{\texteuro}
leur sont restitués sans pénalité.

\subsection{Flux}

Il convient de formaliser l'ensemble des chaînes afin d'en déduire des règles
de gestion qui éviteront la consultation fréquente du bureau et de la direction
et des dysfonctionnements. En effet, le rôle de chaque opérateur est souvent
mal défini. Ce fut mis en évidence à de nombreuses reprises lors des entretiens,
les interrogés devant se référer à d'autres interlocuteurs.

Voir l'annexe \ref{flux}.