\section*{Introduction}

\textit{«~Quel système d'information pour une structure à la gestion
étudiante et bénévole ?~»}

Cet audit, établi dans le cadre du cours de Management des Sytèmes
d'Information de M. Spinard à l'Université Joseph Fourier, est issu d'une série
de rencontres avec l'équipe d'Éponyme, présentée en \ref{presentation}.
Le choix de cette entreprise n'est pas anodin. Deux d'entre nous s'y sont
engagés bénévolement, dans le cadre d'un stage, d'un emploi à l'ENE et de la
représentation d'une université. L'un d'entre nous a également coordonné la
conception et le développement du module «~Pépinière~» du nouveau site Web,
aujourd'hui en phase de test.

De services en services, mais également auprès des bénévoles, nous
avons souhaité déterminer quels outils s'intégraient au fonctionnement du
bâtiment, que ce soit dans les missions régies des contrats et conventions,
présentés en \ref{contrats}, ou organisées selon les volontés de l'association.
Il s'est avéré que non seulement ces outils manquaient de fonctionnalité et
d'efficacité, que les informations à la fois redondantes et insuffisantes,
mais également que la structuration, présentée en \ref{structuration}, et les
règles de gestion, présentées en \ref{gestion}, étaient mal ou pas définies,
et très insuffisamment documentées.

Devant le besoin de formalisation identifié avec le bureau de l'association,
nous avons donc choisi d'utiliser également le présent texte comme une source
de données pour prochain document de référence, dépassant les attentes du
cahier des charges de l'audit.
