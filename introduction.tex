\section*{Introduction}

\textit{«~Comment répondre aux attentes de la Délégation de Service Public liées à la Pépinière d'association ?~»}

Cet audit, établi dans le cadre du cours de Management des Sytèmes d'Information de M. Spinard à l'Université Joseph Fourier, est issu d'une série de rencontres avec l'équipe d'Éponyme, présentée en \ref{presentation}. De services en services, mais également auprès des bénévoles, nous avons souhaité déterminer quels outils s'intégraient au fonctionnement du bâtiment, que ce soit dans les missions régies des contrats et conventions, présentés en \ref{contrats}, ou organisées selon les volontés de l'association. Il s'est avéré que non seulement ces outils étaient peu fonctionnels et efficaces, les informations fortement redondantes mais insuffisantes, mais que la structuration, présentée en \ref{structuration} et les règles de gestion, présentées en \ref{gestion}, étaient mal ou pas définies, et pas documentées.

Devant le besoin exprimé de formalisation, nous avons donc choisi d'utiliser ce document comme l'ébauche d'un prochain document de référence pour l'association, dépassant les attentes du cahier des charges de l'audit.